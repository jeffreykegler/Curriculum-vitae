\documentclass[12pt,margin,line]{res}

\usepackage{hyperref}
\hyphenation{data-base}

\setlength{\oddsidemargin}{-0.5in}
\setlength{\evensidemargin}{-0.5in}
\setlength{\textwidth}{6.0in}

\newcommand{\internalskip}{\vspace{-.15in}}
\newenvironment{mylist}{
  \begin{list}{}
  {%
      \setlength{\itemsep}{0in}%
      \setlength{\leftmargin}{0.2in}%
      \setlength{\parsep}{0in}%
      \setlength{\parskip}{0in}%
      \setlength{\partopsep}{0in}%
      \setlength{\topsep}{0in}%
  }
}
{\end{list}}


\begin{document}

\name{Jeffrey Kegler \vspace*{.1in}}
\address{\url{http:://www.jeffreykegler.com/}}

\begin{resume}

\section{\sc Academics}
{\bf Yale University}, New Haven CT
\begin{mylist}
\item[] M.S., Computer Science \hfill  1978
\item[] B.A., Administrative Science \hfill  1975
\end{mylist}

\internalskip
{\bf Yale Medical School}, New Haven CT
\begin{mylist}
\item[] Lecturer \hfill 1979 to 1981\\
\end{mylist}
\vspace{-.2in}

\section{\sc Selected Publications}

``Marpa, a practical general parser: the recognizer'',
March 2012,
\url{https://github.com/downloads/jeffreykegler/Marpa-theory/recce.pdf}.
This paper contains the theory behind the
recognizer portion of the
Marpa algorithm.
It includes proofs of correctness, and of the complexity claims.

{\bf Perl and Undecidability}, a three part series published
in {\bf The Perl Review}:
``The Halting Problem'', Volume 4, Issue 2, Spring 2008, pp. 21-25;
``Rice's Theorem'', Volume 4, Issue 3, Summer 2008, pp. 23-29; and
``Perl Is Undecidable'', Volume 5, Issue 0, Fall 2008, pp. 7-11.
The articles can be found online, with updates and corrections,
at \url{http://www.jeffreykegler.com/Home/perl-and-undecidability}.
According to Wikipedia's ``Programming Languages'' article,
(\url{http://en.wikipedia.org/wiki/Programming_language}),
these contain the first proof of undecidability for the {\bf syntax}
of a programming language.

{\bf The God Proof},
a novel based on Kurt G\"{o}del's
ontological proof of the existence
of God,
available at Amazon:
\url{http://www.amazon.com/God-Proof-Jeffrey-Kegler/dp/1434807355}.

Thirty books in Urban Land Institute's 
{\bf Dollars and Cents of Shopping Centers}
series for 1990, 1993, 1995, 1997 and 1998
(as {\bf co-author}
with ULI).

``A Polynomial Time Generator for Minimal Perfect Hash Functions'',
{\bf Communications of the ACM},
vol.  29, no.  6, pp.  556-557, 1986.

\section{\sc Open Source Software}

{\bf Marpa::XS} \\
\url{https://metacpan.org/module/Marpa::XS} \hfill 2011

\internalskip
Marpa::XS, the first stable release of the Marpa parsing algorithm,
was recognized as one of Perl's most important modules
within weeks of its 1.0 release in December 2011.
The release of Marpa::XS was
pre-announced by National Public Radio's ``NPR Tech Team'' on their
news feed (19 Nov 2011),
and cited by the Perl Foundation as evidence that
``Perl is bigger and better than ever''
(Perl Foundation News, 5 January 2012).
By its 6-month anniversary,
Marpa::XS was a Fedora package (as of release 16),
had received its first academic citation
(DOI: 10.4230/OASIcs.SLATE.2012.41),
and was mentioned in the Perl language's
own documentation (perlfaq4 as of 5.16.0).

{\bf Marpa::R2} \\
\url{https://metacpan.org/module/Marpa::R2} \hfill 2012

\internalskip
Marpa::R2 is the version of the Marpa algorithm under active
development.
Heavily refactored,
it cleanly separates its Perl upper layers
from the core Marpa algorithm, which is a C library.

{\bf Libmarpa} \\
\url{https://github.com/jeffreykegler/Marpa--R2} \hfill 2012

\internalskip
The library implementing the core of the Marpa algorithm,
written in CWEB, Don Knuth's ``literate programming'' version of 
the C language.
Since
Libmarpa's test and installation environment is in Perl,
Libmarpa is delivered as part of Marpa::R2.

\section{\sc Professional Experience {\footnotesize to 1984}}

{\bf Applied Materials} \\
{\em Senior Sysadmin} \hfill 2005 to 2007

\internalskip
Drove Sarbanes-Oxley remediation in Applied's 400+ server IT shop
(mixed Sun and HP).
Translated requirements from Accounting and Security into operator-actionable
procedures, obtained sign-off from Accounting and Security that they in
fact regarded adherence to these procedures as SOX compliance, and tracked
and coordinated follow-through of the procedures.

{\bf Sun Microsystems} \\
{\em IT Consultant} \hfill 1997 to 2001

\internalskip
{\bf\footnotesize Engagement Architecture}:
In Sun's small Custom Engineering group, sorted and routed raw leads.
On the productive leads, determined the customer's requirements, and from
these established scope, customer responsibilities, deliverables, acceptance
criteria and level of effort.
Explained and sold these and the resulting price to the customer and closed
the deal.
Carried out many of the engagements, with Sun billing the work on an hourly
basis.

\internalskip
{\bf\footnotesize Solaris Device Driver Development}:
Custom Engineering team's technical lead, with a specialty in device drivers
and streams modules.
Wrote device driver for Quatech and another on internal project for Sun.
Other clients wrote the software themselves, calling on me when they had
questions or hit difficulties.
These ``technical assistance''
clients included Lockheed Martin, General Dynamics, Integrated Telecom
Solutions, and Lucent.

\internalskip
{\footnotesize\bf Solaris Systems Software Development}:
Wrote Solaris systems software for the Idaho State Lottery, Motorola,
Telefonica and Colegio de Registradores (the last two in Spain).
Provided technical assistance on the writing of systems software to Riolabs,
TCSI and IBM.

\internalskip
Performed a C++ compiler enhancement for Silicon Metrics.
Provided technical assistance on application software to Urban Land Institute.

{\bf Algorists, Inc.} \\
{\em President and Owner} \hfill 1984 to 1998

\internalskip
{\footnotesize\bf Technical Sales}:
For Algorists' entire history, full responsibility for sales cycle.

\internalskip
{\footnotesize\bf Management}:
At various times, responsible for management of other consultants.
These included partners within Algorists, and consultants hired by Algorists
or its clients.

\internalskip
{\footnotesize\bf Email Server Development}:
For Sun Microsystems, developed a large distributed email server using
 sendmail and wrappers written in C.


\internalskip
{\footnotesize\bf UNIX Device Driver Development}:
Developed device drivers for clients ICL Datachecker, SCOPE, NOVA, Contel,
KSE and ACS.

\internalskip
{\footnotesize\bf UNIX Systems Programming}:
UNIX systems programming for clients Hewlett Packard, Applied Materials,
ICL Datachecker, ITM, Telenet and AT\&T.
Also for HP, HP/UX internals.

\internalskip
{\footnotesize\bf Other}:
For Sun Microsystems, diagnosed network throughput issues, and installed
firewalls.
Developed firmware for ICL Datachecker and SAIC.
Developed application and database software for Urban Land Institute, Kendrick
and Company, the National Association of Realtors and TYX.
Consulted at the Federal Aviation Administration.

\end{resume}
\end{document}
